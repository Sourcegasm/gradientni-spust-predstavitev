\subsection*{Ekliptični koordinatni sistem}
\label{eklipticni_sistem}
	
\paragraph{}
Eklipti"cni koordinatni sistem je eden izmed koordinatnih sistemov za določanje lege nebesnih teles. Lokacijo se dolo"ca glede na sonce. Tako kot imamo za dolo"canje lokacije na zemlji zemljepisno "sirino in zemljepisno dol"zino, imamo pri eklipti"cnem koordinatnem sistemu eklipti"cna dol"zino in eklipti"cno "sirino, ki pomenita isto, samo da namesto to"cke na zemljinem povr"sju definirata to"cko na povr"sju sonca. Predstavljamo si, da smo na Soncu (seveda pono"ci, da ni prevro"ce) in za dolo"canje koordinat z latitudo in longitudo naredimo isto kot na Zemlji, samo da je tokrat Sonce na"sa Zemlja. Za dodatek imamo "se tretji podatek, to je razdalja od Sonca. Da povzamemo:
\begin{description}
	\item[$l$] longtituda, ekliptična dolžina, od $0^\circ$ do $360^\circ$.
	\item[$b$] latituda, ekliptična širina od $-90^\circ$ do $90^\circ$.
	\item[$r$] razdalja
\end{description}

Eklipti"cna "sirina je definirana egoisti"cno, tako da je ravnina na kateri le"zi zemljina orbita, vedno 0º latitude.

Eklipti"cna "sirina je definirana glede na neko son"sno pego.

\paragraph{}
V kartezične koordinate lahko podatke pretvorimo na podoben na"cin kot polarne koordinate pretvarjamo v kartezi"cne. Samo da imamo tokrat neke vrste 3D polarni zapis. Na"se formule so:\newpage
$$x = r \cos b \cos l$$
$$y = r \cos b \sin l$$
$$z = r \sin b$$

\paragraph{}
Te koordinate "zelimo sedaj pretvoriti iz 3D v 2D koordinatni sistem, da jih bomo lahko uporabili v ena"cbi za sto"znice. To bi lahko naredili tako, da bi vzeli projekcijo na"sih to"ck na ravnino. Ker pa tiri ve"cine planetov v na"sem oson"cju ležijo v (skoraj) isti ravnini, se bomo naredili fizike in preprosto zanemarili $z$ koordianto.