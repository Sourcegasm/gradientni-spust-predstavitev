\paragraph{}
Pri funkciji, ki je odvisna od ve"cih parametrov se zakomplicira tudi odvod. Tako funkcijo moramo odvajati po vsaki spremenljivki posebej, kar pomeni da potrebujemo funkcijo $f(x,y)$ odvajati po $x$ in po $y$. S tem dobimo dva odvoda. Odvod po $x$ nam pove kako funkcija nara"s"ca oziroma po $x$ osi, odvod po $y$ pa nam seveda pove kako funkcija nara"s"ca oziroma pada po $y$ osi. Takemu odvodu re"cemo delni odvod, zapi"semo pa ga kot $\frac{\partial f(x,y)}{\partial x}$ za delni odvod po $x$ in $\frac{\partial f(x,y)}{\partial y}$ za delni odvod po $y$.

\paragraph{}
Za trenutek se vrnimo na za"cetni problem, ki nas je pripeljal do sem. "Zelelimo najti tako premico, ki se najbolj prilega vsem danim to"ckam. Definirali smo "ze skupno napako in ugotovili da je na"sa napaka funkcija, ki je odvisna od spremenjivk $a$ in $b$. Sedaj lahko pora"cunamo, kako strma je na"sa napaka za dani $a$ in $b$ po parametru $a$ in kako strma je po parametru $b$. Druga"ce povedano, izra"cunamo lahko delni odvod napake po $a$ in delni odvod napake po $b$.
$$\frac{\partial \varepsilon}{\partial a} =
\sum_{i=1}^{N} 2 \frac{\partial (a x_i + b - y_i)}{\partial a} =
2 \sum_{i=1}^{N} (a x_i + b - y_i)x_i$$

$$\frac{\partial \varepsilon}{\partial b} =
\sum_{i=1}^{N} 2 \frac{\partial (a x_i + b - y_i)}{\partial b} =
2 \sum_{i=1}^{N} (a x_i + b - y_i)\cdot1 = 2 \sum_{i=1}^{N} (a x_i + b - y_i)$$