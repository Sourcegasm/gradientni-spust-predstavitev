\paragraph{}
Gradient neke funkcije je vektor, ki ka"ze v smer nara"s"canja te funkcije. Gradient funkcije $f(x)$ ozna"cimo z $\nabla f(x)$. Pri funkciji z enim parametrom ($f(x)$), ima gradient te funkcije samo eno komponento, to je odvod funkcije $f(x)$. To zapi"semo kot:

\[\nabla f(x) = \begin{pmatrix}f'(x)\end{pmatrix} \]

\paragraph{}
"Ce je funkcija odvisna od dveh spremenljivk, je vsaka komponenta gradienta en delni odvod. To pomeni da nam prva komponenta gradienta pove kako se funkcija spreminja po $x$ osi, druga komponenta gradienta pa nam pove kako se funkcija spreminja po $y$ osi. Gradient funkcije $f(x,y)$ lahko torej zapi"semo kot:

\[\nabla f(x,y) = \begin{pmatrix}
\frac{\partial f(x,y)}{\partial x} (x) &
\frac{\partial f(x,y)}{\partial y} (y)
\end{pmatrix} \]

\paragraph{}
V zgornjih dveh primerih smo zapisali gradient funkcije v dolo"ceni to"cki $x$ oziroma $(x, y)$. "Ce bi drugi primer zapisali v spl"snem bi to zgledalo kot:

\[\nabla f = \begin{pmatrix}
\frac{\partial f}{\partial x} &
\frac{\partial f}{\partial y}
\end{pmatrix}
\]

To pomeni da lahko funkcijo odvisno od $n$ paramtertov, zapi"semo kot:
\[\nabla f = \begin{pmatrix}
\frac{\partial f}{\partial x_1} &
\frac{\partial f}{\partial x_2} &
\frac{\partial f}{\partial x_2} &
\dots &
\frac{\partial f}{\partial x_n}
\end{pmatrix} \]