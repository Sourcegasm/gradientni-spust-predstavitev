\paragraph{}
Glavna razlika med logistično in linearno regesijo je, da pri linearni regresiji določamo zvezno spremenljivko ($y$ je odvisen od $x$), medtem ko nam pri logistični regresiji model vrne kakšna je verjetnost, da vhodni podatki sodijo v določeno kategorijo.

\paragraph{}
To si lahko predstavljamo kot funkcijo, ki vrne ve"c parametrov. Linearna regresija nam predstavlja funkcijo, ki vrne eno "stevilko oziroma parameter. Logisti"cna regresija je funkcija, ki vzame ve"c parametrov in vrne ve"c parametrov, to je "stevilk.

\paragraph{}
Logisti"cna regresija se uporablja za klasifikacijo. Eden izmed takih primerov je prepoznavanje "stevil na sliki. "Ce se lotimo prepoznavanja enomestnih "stevil, bi nam logisti"cna regresija vrnila 10 parametrov oziroma "stevil. Vsako izmed teh "stevil nam predstavlja verjetnost, da je na sliki dolo"cena cifra. Prvo "stevilo nam recimo predstavlja verjetnost, da je na sliki 0, drugo da je na sliki 1, tretje da je na sliki 2 in tako naprej.
